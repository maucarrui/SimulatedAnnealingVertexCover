\documentclass{article}

% Márgenes.
\usepackage{geometry}
\addtolength{\hoffset}{-0.2cm}
\addtolength{\textwidth}{0.4cm}
\addtolength{\voffset}{-0.5cm}
\addtolength{\textheight}{1cm}

% Caracteres especiales del español.
\usepackage[utf8]{inputenc}

% Para el lenguaje español.
\usepackage[spanish]{babel}

% Símbolos matemáticos
\usepackage{amssymb}

% Binomio de Newton.
\usepackage{amsmath}

% Imágenes
\usepackage{graphicx}
\graphicspath{ {./img/} }

% Algoritmos
\usepackage[ruled,vlined,linesnumbered]{algorithm2e}

% Para que los figuras se queden en su lugar.
\usepackage{float}

% Macros.
\newcommand{\tbf}[1]{\textbf{#1}}

\newcommand{\tit}[1]{\textit{#1}}

\newcommand{\ttt}[1]{\texttt{#1}}

\newcommand{\np}{\mathcal{NP}}

% Título y autor del reporte.
\title{Uso de la Heurística de Aceptación por Umbrales para Resolver
       el Problema de la Cubierta de Vertices (Vertex Cover)}
\author{Carrasco-Ruiz Mauricio  \\
	maucarrui@ciencias.unam.mx  \\
        Universidad Nacional Autónoma de México (UNAM) \\
        Facultad de Ciencias \\
        Heurísiticas de Optimización Combinatoria \\
	}

% Fecha de hoy.
\date{\today} 

\begin{document}
  \maketitle

  \begin{abstract}
    Dada una gráfica $G = (V, E)$, el problema de la cubierta 
    de vértices es un problema de optimización combinatoria en 
    donde se busca encontrar el mínimo conjunto de vértices $X$
    de tal forma que cualquier arista en la gráfica sea adyacente
    a uno de los vértices en $X$. Se optó por utilizar la heurística
    de aceptación por umbrales para resolver este problema, y su 
    implementación fue llevada a cabo en C++. La experimentación
    se hizo sobre 4 gráficas planas con diferentes cantidades de 
    vértices y aristas cada una. Y para cada una de las gráficas 
    la heurística logro encontrar un conjunto mínimo que cubriera 
    a toda la gráfica. Concluyendo así que la huerística escogida 
    es una buena alternativa para poder resolver el problema de
    la cubierta de vértices.
  \end{abstract}

  \section{Introducción} \label{intro}
  
  \subsection{El Problema de la Cubierta de Vértices (Vertex Cover)}
  Sea $G = (V, E)=$ una gráfica en donde $V$ es el conjunto de 
  vértices y $E$ el conjunto de aristas. \\

  \tbf{Definición:} Sea $X \subseteq V$, se dice que $X$ es una 
  cubierta si y sólo si para cada $uv \in E$ se tiene que 
  $u \in X$ ó $v \in X$. \\

  El problema de la cubierta de vértices, como su nombre lo dice, 
  consiste en encontrar esta cubierta de vértices. Su versión de 
  decisión de este problema es famoso por ser uno de los 21 
  problemas $\np$-completos de Karp. Sin embargo la versión que nos 
  interesa para este trabajo es su versión de optimización, la 
  cual consiste en encontrar la mínima cubierta de vértices de una
  gráfica; esta versión se encuentra en la clase $\np$-duro.

  Al ser un problema $\np$-duro, no se si sabe si existe o no 
  un algoritmo que lo pueda resolver en tiempo polinomial, por ello
  se han desarrollado una gran cantidad de algoritmos y heurísticas 
  que intentan dar una proximación a la solución de este problema. 
  Entre los más famosos se encuentra el algoritmo glotón
  que logra encontrar una cubierta pero no asegura que sea la mínima,
  y lo mismo pasa con muchas otras huerísticas.

  Para este trabajo se decidio trabajar con una variante del 
  recocido simulado: la aceptación por umbrales. Y ver qué tan
  buena alternativa es para resolver el problema de la cubierta
  de vértices. 

  Cabe mencionar que las gráficas sobre las que se trabajaron fueron 
  gráficas planas, pues es muy fácil que se tome un conjunto de vértices
  y se observe que hay un número elevado de aristas que son cubiertas por 
  múltiples vértices de la cubierta; resultando en un reto encontrar 
  la cubierta de tal forma que el número de aristas repetidas sea el 
  menor posible.

  \subsection{Aceptación por Umbrales (Threshold Acceptance)}
  Para que podamos abordar sobre la heurística de Aceptación por Umbrales,
  es necesario que primero hablemos sobre el Recocido Simulado.

  La heurística del Recocido Simulado fue propuesa en 1983 por
  Scott Kirkpatrick, Daniel Gelatt y Mario Vecchi. Como su nombre
  lo indica, se trata de simular la técnica de \tit{recocido}
  utilizada en la metalurgía para deformar y fortalezer metales sin 
  que esta manipulación presente un defecto sobre ellos. Esto lo 
  consiguen primero calentando el metal a altas temperaturas,
  esperando a que se enfríe un poco, lo deforman y se repite hasta
  que se obtenga la forma deseada.

  Así, dentro de la computación, el objetivo del Recocido Simulado
  es ``calentar'' una solución e irla deformando poco a poco 
  mediante el enfriamiento hasta que la solución que nos quede 
  sea factible y lo \tit{suficientemente} buena.

  La Aceptación por Umbrales, propuesta en 1990 por Gunter Duek y
  Tobias Scheuer, heredó toda la metodología del Recocido
  Simulado y además agrego una nueva forma de aceptar o rechazar
  soluciones dependiendo de si estas superaban o no un umbral.
  Los parametros de la huerística son una temperatura initial $T$,
  una solución inicial $s$, una función $N: S \rightarrow S$ que
  recibe una solución y regresa a un vecino de esta solución, 
  un factor de enfriamiento $\phi$, y una función de costo 
  $f: S \rightarrow \mathbb{R}$ que recibe una solución y regresa
  su costo.

  En un principio, la idea de la huerística es generar un lote 
  de vecinos de $s$, y evaluar para cada vecino $s'$ si se 
  cumple que $f(s') < f(s) + T$. Si para alguno de ellos se cumple
  lo anterior, entonces descartamos a $s$ y nuestra solución actual 
  será $s'$; se dice entonces que $s'$ fue aceptada. Después
  multiplicamos a $\phi$ por $T$ y esta será nuestra nueva 
  temperatura. Y repetimos este procedimiento hasta que 
  $T < \varepsilon$ o se tenga un limite para la cantidad de veces 
  que se van a repetir los pasos anteriores.

  \section{Desarrollo} \label{development}

  \section{Experimentación y Resultados} \label{results}

  \section{Discusión} \label{discussion}

  \section{Conclusión} \label{conclussion}
  
\end{document}
